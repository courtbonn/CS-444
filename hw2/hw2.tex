\documentclass[letterpaper,10pt,draftclsnofoot,onecolumn,titlepage]{IEEEtran}

\usepackage{graphicx}
\usepackage{amssymb}
\usepackage{amsmath}
\usepackage{amsthm}
\usepackage{alltt}
\usepackage{float}
\usepackage{color}
\usepackage{url}
\usepackage{enumitem}
%\usepackage{pstricks, pst-node}
\usepackage{geometry}
\usepackage{bookmark}

\geometry{margin = .75in}

\usepackage{hyperref}


\def\name{Courtney Bonn, Isaac Chan}
\def\grp{Group \#39}

\hypersetup{
	colorlinks = true,
	urlcolor = black,
	pdfauthor = {\name},
	pdftitle = {CS 444, Project 2},
	pdfsubject = {CS 444, Project 2},
	pdfpagemode = UseNone
}

\begin{document}

\title{Project 2: I/O Elevators}
\author{\name \\ \grp}

\maketitle

\begin{abstract}
Abstract here.
\end{abstract}

\clearpage
\section{Design Choice}
    
\section{Version Control Log}
%\begin{tabular}{l l l}\textbf{Detail} & \textbf{Author} & \textbf{Description}\\\hline
\href{https://github.com/courtbonn/CS-444/commit/29547f64718379356bd13263071a531bdd0896b5}{29547f6} & Courtney Bonn & Added Project 1 Folder\\\hline
\href{https://github.com/courtbonn/CS-444/commit/683a101800f08d0bcd4492d65d8829d25f488f21}{683a101} & Courtney Bonn & Revert "Added Project 1 Folder"\\\hline
\href{https://github.com/courtbonn/CS-444/commit/e794b6c0aad9a13a17177922b6ff415d677495b5}{e794b6c} & Anonymous & Project created\\\hline
\href{https://github.com/courtbonn/CS-444/commit/23bfd89f069e99232fe59b7730f7625cbf29fa44}{23bfd89} & Courtney LeeAnne Bonn & Update on Overleaf.\\\hline
\href{https://github.com/courtbonn/CS-444/commit/f9ef6fd2b923276b21d789c4caba0985800d1515}{f9ef6fd} & Courtney Bonn & Added Project 1 LaTex files\\\hline
\href{https://github.com/courtbonn/CS-444/commit/5d11dbfd983f4c3f1911da1797a2be0fae8bf2cf}{5d11dbf} & Courtney Bonn & Merge branch 'master' of https://git.overleaf.com/11476102qmhgtdkzzwrz\\\hline
\href{https://github.com/courtbonn/CS-444/commit/c96843d51fef0558071c08e6256c3020fd84a999}{c96843d} & Courtney LeeAnne Bonn & Update on Overleaf.\\\hline
\href{https://github.com/courtbonn/CS-444/commit/d7d8486bd31847448e59f0803e792d6ea53e516a}{d7d8486} & Courtney Bonn & Merge branch 'master' of https://git.overleaf.com/11476102qmhgtdkzzwrz\\\hline
\href{https://github.com/courtbonn/CS-444/commit/768a313b027c8be433190f3d92ee1d3f8d69daad}{768a313} & Courtney Bonn & removed overleaf folder\\\hline
\href{https://github.com/courtbonn/CS-444/commit/431243b6da2468a84fa6ef1b39b6490ea81620a0}{431243b} & Courtney Bonn & removed additional files\\\hline
\href{https://github.com/courtbonn/CS-444/commit/ab0f559ed2890ba06230c459c2b1b96a8f69ac6a}{ab0f559} & Courtney Bonn & Added Project 1 Folder\\\hline
\href{https://github.com/courtbonn/CS-444/commit/3f97e0af2ade6dd51b96d8d9cfd751fb70deff0d}{3f97e0a} & Courtney LeeAnne Bonn & Update on Overleaf.\\\hline
\href{https://github.com/courtbonn/CS-444/commit/91329b56b14c2eb1d1058f3f3275215813b83e6c}{91329b5} & Courtney Bonn & Merge branch 'master' of https://git.overleaf.com/11476102qmhgtdkzzwrz\\\hline
\href{https://github.com/courtbonn/CS-444/commit/8d79a3663a74c19792bf0bcaf63e81d6c8ed0202}{8d79a36} & Isaac Chan & add base concurrency assignment\\\hline
\href{https://github.com/courtbonn/CS-444/commit/2986ade21091b19b494a0ab684c209f17043f11c}{2986ade} & Courtney LeeAnne Bonn & Update on Overleaf.\\\hline
\href{https://github.com/courtbonn/CS-444/commit/aad3085e0ed73b8dc15ba16c44587b6216d605f4}{aad3085} & Courtney Bonn & Merge branch 'master' of https://git.overleaf.com/11476102qmhgtdkzzwrz\\\hline
\href{https://github.com/courtbonn/CS-444/commit/c91c8019f4ca95febb45ce0d5e767b08ca0d5edc}{c91c801} & Courtney Bonn & Removed overleaf files\\\hline
\href{https://github.com/courtbonn/CS-444/commit/83cdd06b6f86e488b8cb42351ff1ffe0f6f14bc9}{83cdd06} & Courtney Bonn & Added LaTex Overleaf File\\\hline
\href{https://github.com/courtbonn/CS-444/commit/26a943d14694a3da4a741850230e552ed686ffb1}{26a943d} & Isaac Chan & finalize RDRAND support\\\hline
\href{https://github.com/courtbonn/CS-444/commit/783432f7223bc4f0398a94de020115e8cbd18e8c}{783432f} & Isaac Chan & add todo notes for methods\\\hline
\href{https://github.com/courtbonn/CS-444/commit/fd6aa5dac056c23cfbb9cd182e1ae7190d0124a9}{fd6aa5d} & Courtney Bonn & added latex file\\\hline
\href{https://github.com/courtbonn/CS-444/commit/ae045f44596599c8b89804b6114574eeed76b664}{ae045f4} & Courtney Bonn & Updated hw1.tex\\\hline
\href{https://github.com/courtbonn/CS-444/commit/e35c2615d97f2625ebaa1e5845d799b49457d7b0}{e35c261} & Courtney Bonn & Updated hw1.tex\\\hline
\href{https://github.com/courtbonn/CS-444/commit/6de87295369d96c5261e6e729094806ad5c14014}{6de8729} & Courtney Bonn & Added pdf\\\hline
\href{https://github.com/courtbonn/CS-444/commit/30cc64929a330c8dea22b408f6f905ab82c5f9b8}{30cc649} & Courtney Bonn & deleted unnecessary folder\\\hline
\href{https://github.com/courtbonn/CS-444/commit/7567b06cd1bb85644d12c5c6f20e01439ac303b2}{7567b06} & Courtney Bonn & Moved files into correct folder\\\hline
\href{https://github.com/courtbonn/CS-444/commit/b27b47d9c1ae89a8396ace289ae123d3bdd30ea8}{b27b47d} & Courtney Bonn & deleted extra folder\\\hline
\href{https://github.com/courtbonn/CS-444/commit/65f44e9b825be2bfaf5fb62037531ac516fe91d3}{65f44e9} & Courtney Bonn & Added qemu flag explanations\\\hline
\href{https://github.com/courtbonn/CS-444/commit/0e5a2a8615574269544771707abf67ed5f775a46}{0e5a2a8} & Courtney Bonn & Added work log and bib\\\hline
\href{https://github.com/courtbonn/CS-444/commit/bc81e0c881560deeef263a0732dae8476999999d}{bc81e0c} & Courtney Bonn & updated PDF\\\hline
\href{https://github.com/courtbonn/CS-444/commit/28602d145c5249d1feb8a2813a429ddc43211c2c}{28602d1} & Isaac Chan & add mutexes and rand time to repo master\\\hline
\href{https://github.com/courtbonn/CS-444/commit/eba890e8ea9640c4723e587f8aedafa088e14bd9}{eba890e} & Isaac Chan & added the rest of hw1.c currently with seg fault\\\hline
\href{https://github.com/courtbonn/CS-444/commit/f64315399021183e4c4e9ddaa2b4f55bf24ca3f0}{f643153} & Courtney Bonn & fixed seg fault; program still not running correctly\\\hline
\href{https://github.com/courtbonn/CS-444/commit/223b3969975954043e29b636d3047910f5b09a26}{223b396} & Isaac Chan & add Isaac's worklog and part of concurrency write up.\\\hline
\href{https://github.com/courtbonn/CS-444/commit/59a039f3dc01a5f965dda35aebac82b56188ecc7}{59a039f} & Courtney Bonn & Added some lines to try and fix issue; not working still\\\hline
\href{https://github.com/courtbonn/CS-444/commit/a1571b350bf3104b5add44e6ce48917ca888dea6}{a1571b3} & Courtney Bonn & finished producer-consumer problem; no errors\\\hline
\href{https://github.com/courtbonn/CS-444/commit/cb9a9c988fdb22f5364a44ecfc95c18a3f72ae3d}{cb9a9c9} & Courtney Bonn & Added abstract, updated worklogs, added concurrency answers, fixed makefile\\\hline
\href{https://github.com/courtbonn/CS-444/commit/d2cd6dfd0888c09678fff0cb0b561ad5f65cfb9b}{d2cd6df} & Courtney Bonn & fixed pdf\\\hline\end{tabular}


\section{Work Log}
\begin{center}
\begin{tabular}{p{3cm}p{1cm}p{1cm}p{10cm} }
 Date  & Time & Person & Event \\ \hline
October 25, 2017 & 5:15pm & Courtney & Began Project 2 LaTeX file \\
		 & 6:15pm & Courtney & Changes Tex Template to match Project 2 and made sure it compiled correctly with "make" \\
October 29, 2017 & 11:30am & Courtney & Started reading Chapter 14 of Love's Kernel book \\                 
		 & 11:45am & Courtney & Copied noop-iosched.c to our working directory, renamed it sstf-iosched.c, and changed instances of noop to look \\ 
		 & 12:00pm & Courtney & Continued research on NOOP, LOOK, and C-LOOK algorithms \\
\end{tabular}
\end{center}

\section{Write Up}
\begin{enumerate}
                \item What do you think the main point of this assignment is? \\
The main point of this assignment is to learn how to work with the disk scheduler, or I/O scheduler, on the kernel. Using the current I/O schedulers as base algorithms to work off of, we will build a LOOK or CLOOK algorithm on the NOOP scheduler. 
                \item How did you personally approach the problem? Design decisions, algorithm, etc. \\
This assignment took a lot of research ahead of time before actually sitting down to work on the code. 
First off, we needed to understand what the current algorithm was doing. 
NOOP is essentially a first-come-first-served (FCFS) or FIFO algorithm, meaning whatever request it receives first is the first one it is going to process. 
Additionally, it does not do any sorting. It does sort a new request with adjacent requests. 

LOOK or CLOOK algorithm description here -- depending on which one we choose. 

To implement our selected algorithm, we made use of the Kernel's Linked List implementation. 
                \item How did you ensure your solution was correct? Testing details, for instance.
                \item What did you learn?
                \item How should the TA evaluate your work? Provide detailed steps to prove correctness 
\end{enumerate}



\end{document}
