\documentclass[letterpaper,10pt,draftclsnofoot,onecolumn,titlepage]{IEEEtran}

\usepackage{graphicx}
\usepackage{amssymb}
\usepackage{amsmath}
\usepackage{amsthm}
\usepackage{alltt}
\usepackage{float}
\usepackage{color}
\usepackage{url}
\usepackage{enumitem}
%\usepackage{pstricks, pst-node}
\usepackage{geometry}
\usepackage{bookmark}

\geometry{margin = .75in}

\usepackage{hyperref}


\def\name{Courtney Bonn, Isaac Chan}
\def\grp{Group \#39}

\hypersetup{
	colorlinks = true,
	urlcolor = black,
	pdfauthor = {\name},
	pdftitle = {CS 444, Project 2},
	pdfsubject = {CS 444, Project 2},
	pdfpagemode = UseNone
}

\begin{document}

\title{Project 2: I/O Elevators}
\author{\name \\ \grp}

\maketitle

\begin{abstract}
For the second kernel assignment, we are tasked with building a new I/O Scheduler that is based off of the current NOOP Scheduler in the Linux Kernel. 
The algorithm that will be used for the new I/O Scheduler is CLOOK. 
\end{abstract}

\clearpage
\section{Design Plan}
The algorithm we chose to implement is CLOOK. After examining the NOOP scheduler, we came to the conclusion that the only two methods to be altered 
are the dispatch and add\_request methods. The dispatch requires an addition to tell if the request is for reading or writing. The add\_request method 
requires an addition to put a new request into the queue. Because the CLOOK algorithm is circular, the request must be put in the correct spot in the 
queue. This can be done by: check the sector position while iterating and then insert the request if the sector is larger. 
    
\section{Version Control Log}
\begin{tabular}{p{2cm} p{3cm} p{12cm}}\textbf{Detail} & \textbf{Author} & \textbf{Description}\\\hline
\href{https://github.com/courtbonn/CS-444/commit/af5d754ea19c6bdc6fb977d874932a62b5176beb}{af5d754} & Courtney Bonn & started hw2 latex file\\\hline
\href{https://github.com/courtbonn/CS-444/commit/fd7b5f2f08a4033193091a2c005324d4e8c83dad}{fd7b5f2} & Courtney Bonn & have basic dining concurrency completed and compiling/running correctly. need to add status of forks and name philosophers\\\hline
\href{https://github.com/courtbonn/CS-444/commit/f0ebac09ef7efebaa007d5d0335d6622f2d72c31}{f0ebac0} & courtbonn & Add files via upload\\\hline
\href{https://github.com/courtbonn/CS-444/commit/c0b2cba2793c2b835c345de95d1adc408d3590a8}{c0b2cba} & Courtney Bonn & got latex template set up, compiled correctly\\\hline
\href{https://github.com/courtbonn/CS-444/commit/e7ef5ade60f7075aff7ea7a35a0a7576ad37fbb0}{e7ef5ad} & Courtney Bonn & added pdf to see if it's working\\\hline
\href{https://github.com/courtbonn/CS-444/commit/3cd66fe985151af1a961c9a1510d90d5906b56b5}{3cd66fe} & Isaac Chan & added philo names\\\hline
\href{https://github.com/courtbonn/CS-444/commit/65f8047ce747c6c81206c1b4b7e687d9eb404386}{65f8047} & Isaac Chan & added fork status\\\hline
\href{https://github.com/courtbonn/CS-444/commit/3f955c4d2095239ff38e446b64420fbad1581329}{3f955c4} & Courtney Bonn & added latex file\\\hline
\href{https://github.com/courtbonn/CS-444/commit/672ddcb03b3772fe388444afa3ef9f497483dab5}{672ddcb} & Isaac Chan & added while loop and sleeps to fix overlapping printing\\\hline
\href{https://github.com/courtbonn/CS-444/commit/4e80e2ae4644e5814c5c5ab02280c36ff4fb7cd1}{4e80e2a} & Isaac Chan & add writeup\\\hline
\href{https://github.com/courtbonn/CS-444/commit/e2eb0dad4d249310f06385038a952f71a68c356a}{e2eb0da} & Courtney Bonn & updated worklog\\\hline
\href{https://github.com/courtbonn/CS-444/commit/49d558d3f3bae0525a168a7cb1f950dc5d4df2c4}{49d558d} & Courtney Bonn & Added copy of noop-iosched and changed it to sstf-iosched.c. Replaced all instances of noop with look\\\hline
\href{https://github.com/courtbonn/CS-444/commit/30e072d1c632099297d1bb209b35dcd3bdb02a25}{30e072d} & Courtney Bonn & Updated worklog, made PDF\\\hline
\href{https://github.com/courtbonn/CS-444/commit/a26c4c0fa92b54d5e3198893ae7ef6e9b5fe139a}{a26c4c0} & Courtney Bonn & Trying to fix table width\\\hline
\href{https://github.com/courtbonn/CS-444/commit/03aace950555c852e351dff046443684678ff246}{03aace9} & Courtney Bonn & still working on worklog table\\\hline
\href{https://github.com/courtbonn/CS-444/commit/9e4a626bb9d41d55f124095a31341da5947303d9}{9e4a626} & Courtney Bonn & Table working now\\\hline
\href{https://github.com/courtbonn/CS-444/commit/ed3d13596e17bb82b25c641cbddec202cbe0e020}{ed3d135} & Courtney Bonn & added names to sstf-iosched.c and started answering write up questions\\\hline
\href{https://github.com/courtbonn/CS-444/commit/f1b033bfdcadb8aa0d9af64fd1d6963bdad423e3}{f1b033b} & Courtney Bonn & updated pdf\\\hline
\href{https://github.com/courtbonn/CS-444/commit/b47399d0a5fbc173d73f4f61c1fb1b70f72ca61a}{b47399d} & Isaac Chan & add dispatch and add\_request logic for clook implementation\\\hline
\href{https://github.com/courtbonn/CS-444/commit/8fe6ee4f10e247f5d23c7d082d700fbc7222bbc8}{8fe6ee4} & Isaac Chan & add design plan and update work log\\\hline
\href{https://github.com/courtbonn/CS-444/commit/c6ed66dde259cdf2e290ae5e6885c93ed571b4e7}{c6ed66d} & Isaac Chan & add additional print msgs for easier testing\\\hline
\href{https://github.com/courtbonn/CS-444/commit/5ce345ceb6730fcccacd41f08dc4b3f99c7bb053}{5ce345c} & Courtney Bonn & Added to worklog\\\hline
\href{https://github.com/courtbonn/CS-444/commit/cc980f65b008bd8d30c7d09f934ac11b45b6adab}{cc980f6} & Courtney Bonn & remove .swp file\\\hline
\href{https://github.com/courtbonn/CS-444/commit/49a7ce4718347c93bcb150d53fc5f71355bee0fe}{49a7ce4} & Courtney Bonn & added to worklog and writeup\\\hline
\href{https://github.com/courtbonn/CS-444/commit/fabb6f781a78d68466cc8c23156d25d426f5e2fd}{fabb6f7} & Courtney Bonn & updated worklog\\\hline
\href{https://github.com/courtbonn/CS-444/commit/5b06ffa743ac23d5ffb0b7a4985f121b9b56e6ca}{5b06ffa} & Courtney Bonn & returning dispatch to it's original form\\\hline
\href{https://github.com/courtbonn/CS-444/commit/09ded8afcdf70a8b47e7035542c62d93e76d51dd}{09ded8a} & Courtney Bonn & Add\_request still causing a kernel panic\\\hline
\href{https://github.com/courtbonn/CS-444/commit/498d37252ce619ec5b677dfd70ea2dd315daae75}{498d372} & Isaac Chan & block for patch generation\\\hline
\href{https://github.com/courtbonn/CS-444/commit/cd6c6269ddabeed3e7a761cc42041564bff86d4f}{cd6c626} & Courtney Bonn & finished writeup\\\hline
\href{https://github.com/courtbonn/CS-444/commit/96174f8ba4be80b2cb15a629ad468998537600ea}{96174f8} & Isaac Chan & add unchanged block dir for patch gen\\\hline
\href{https://github.com/courtbonn/CS-444/commit/8d8bb56c6b4e6287ad20283cba4560f59e3e21dd}{8d8bb56} & Isaac Chan & added linux.patch\\\hline
\href{https://github.com/courtbonn/CS-444/commit/193653232e83d3b72dab544abc48d95c78bb2e2c}{1936532} & Courtney Bonn & added correct linux patch\\\hline
\href{https://github.com/courtbonn/CS-444/commit/c7e3b36a90dc324a0bd24af5bbebe27b8f2cd352}{c7e3b36} & Courtney Bonn & removed linux patch\\\hline
\href{https://github.com/courtbonn/CS-444/commit/119116a604998aba96ad0479b6309bbd25d10019}{119116a} & Isaac Chan & rename noop file in unchanged\\\hline
\href{https://github.com/courtbonn/CS-444/commit/ba5aeb22ab9d21ca25fc05605a65288d43e52762}{ba5aeb2} & Isaac Chan & remove unchanged block\\\hline\end{tabular}
\section{Work Log}
\begin{center}
\begin{tabular}{p{3cm}p{1cm}p{1cm}p{10cm} }
 Date  & Time & Person & Event \\ \hline
October 25, 2017 & 5:15pm & Courtney & Began Project 2 LaTeX file \\
		 & 6:15pm & Courtney & Changes Tex Template to match Project 2 and made sure it compiled correctly with "make" \\
October 29, 2017 & 11:30am & Courtney & Started reading Chapter 14 of Love's Kernel book \\                 
		 & 11:45am & Courtney & Copied noop-iosched.c to our working directory, renamed it sstf-iosched.c, and changed instances of noop to look \\ 
		 & 12:00pm & Courtney & Continued research on NOOP, LOOK, and C-LOOK algorithms \\
		 & 10:00pm & Courtney & Working on getting the VM run with the NOOP Scheduler \\
   		 & 10:30pm & Isaac & Examine NOOP scheduler and understand how it works \\
   		 & 11:30pm & Isaac & Design a plan to implement CLOOK algorithm \\
   		 & 11:45pm & Isaac & Begin implementation of CLOOK algorithm \\
October 30, 2017 & 12:30am & Isaac & Complete initial CLOOK implementation \\
		 & 1:00am & Courtney & Working on Kernel error \\
		 & 1:51am & Isaac & Recopied Yocto directory with changed files to rebuild the kernel \\
		 & 1:55am & Isaac & Successfully built the kernel with the new scheduler \\
		 & 2:05am & Courtney & Ran qemu with new scheduler and confirmed that clook scheduler is the default in the VM \\
   		 & 2:20am & Courtney & Design a plan to test the new scheduler \\
		 & 3:45am & Courtney & Still working on how to test scheduler \\
		 & 4:47am & Courtney and Isaac & Trying to figure out why CLOOK scheduler isn't printing/working, not having much success \\
		 & 5:30pm & Courtney and Isaac & Working through Kernel Panic, trying to determine where the error is \\
   		 & 7:30pm & Courtney and Isaac & Wrapping up assignment with final commits and documenting test plan \\
\end{tabular}
\end{center}

\section{Write Up}
\begin{enumerate}
                \item What do you think the main point of this assignment is? \\
The main point of this assignment is to learn how to work with the disk scheduler, or I/O scheduler, on the kernel. 
Using the current I/O schedulers as base algorithms to work off of, we will build a CLOOK algorithm on the NOOP scheduler. 
                \item How did you personally approach the problem? Design decisions, algorithm, etc. \\
This assignment took a lot of research ahead of time before actually sitting down to work on the code. 
First off, we needed to understand what the current algorithm was doing. 
NOOP is essentially a first-come-first-served (FCFS) or FIFO algorithm, meaning whatever request it receives first is the first one it is going to process. 
Additionally, it does not do any sorting. It does sort a new request with adjacent requests. 

CLOOK is a circular variant of LOOK. It only scans in one direction and starts at the beginning when it reaches the end. When a new request arrives it must 
be sorted to the correct spot in the queue.

To implement our selected algorithm, we made use of the Kernel's Linked List implementation. 
                \item How did you ensure your solution was correct? Testing details, for instance.
Our plan for testing was to run a python script that created a file and wrote to it. 
This triggers file/IO steps that we can then check the system log for clook printouts. 
However, because we were unsuccessful at getting the kernel to boot with our new I/O scheduler, we were unable to fully test the solution. 
Once the CLOOK algorithm was written, we were able to select it as the default scheduler when building the kernel and the kernel was able to compile.
We made the necessary adjustments to the qemu command by disabling virtio and changing \textit{root=/dev/vda} to \textit{root=/dev/hda}; however, when the qemu command was ran, there was a kernel panic. 
We attempted for several hours to adjust our algorithm in order to avoid a kernel panic, but we were unable to determine the root of the problem. 
We believe the problem was with our add request function, as that ended up being the only one we modified. 
During the kernel boot, you can see "CLOOK add 2" which indicates that an add does occur, but immediately after an error occurs stating \textit{BUG: unable to handle kernel NULL pointer dereference at 0000008a}. 
We are not 100 percent sure that this is the issue with the kernel panic, but we believe it may be attributed to it. 

                \item What did you learn?
We learned about implementing an elevator algorithm and also with configuring a Linux kernel. We were able to change qemu flags to fit our needs.
Even though we were not able to successfully apply our new scheduler and were unable to determine its correctness, we were able to learn quite a lot with regards to the Linux kernel. 

                \item How should the TA evaluate your work? Provide detailed steps to prove correctness 
			\begin{itemize}
			\item Apply kernel patch file and build the kernel with \textit{make -j4 all}.
			\item The build should ask you what scheduler to use; select CLOOK as the default.
			\item Source the environment script.
			\item Launch qemu with \textit{qemu-system-i386 -gdb tcp::5539 -S -nographic -kernel ./linux-yocto-3.19/arch/x86/boot/bzImage -drive file=core-image-lsb-sdk-qemux86.ext4,if=ide -enable-kvm -usb -localtime --no-reboot --append "root=/dev/hda rw console=ttyS0 debug"}
			\item Connect to qemu with gdb \textit{target remote:5539} and \textit{continue}
			\item There will be many CLOOK messages as the kernel boots.
			\end{itemize}
\end{enumerate}



\end{document}
