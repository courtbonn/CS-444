ss[letterpaper,10pt,draftclsnofoot,onecolumn,titlepage]{IEEEtran}

\usepackage{graphicx}
\usepackage{amssymb}
\usepackage{amsmath}
\usepackage{amsthm}
\usepackage{alltt}
\usepackage{float}
\usepackage{color}
\usepackage{url}
\usepackage{enumitem}
%\usepackage{pstricks, pst-node}
\usepackage{geometry}
\usepackage{bookmark}

\geometry{margin = .75in}

\usepackage{hyperref}

\newcommand*{\signature}[1]{%
	\par\noindent\makebox[3.5in]{\hrulefill} \hfill\makebox[3.0in]{\hrulefill}%
	\par\noindent\makebox[3.5in][l]{#1}	    \hfill\makebox[3.0in][l]{Date}%
}%

\def\name{Courtney Bonn, Isaac Chan}
\def\grp{Group \#39}

\hypersetup{
	colorlinks = true,
	urlcolor = black,
	pdfauthor = {\name},
	pdftitle = {CS461 Problem Statement},
	pdfsubject = {CS461 Problem Statement},
	pdfpagemode = UseNone
}




%pull in the necessary preamble matter for pygments output
%\input{pygments.tex}


\begin{document}

\title{Project 2: I/O Elevators}
\author{\name \\ \grp}

\maketitle

\begin{abstract}
\end{abstract}

\clearpage

%input the pygmentized output of mt19937ar.c, using a (hopefully) unique name
%this file only exists at compile time. Feel free to change that.


\section{Design Choice}
    
\section{Version Control Log}
\begin{tabular}{l l l}\textbf{Detail} & \textbf{Author} & \textbf{Description}\\\hline
\href{https://github.com/courtbonn/CS-444/commit/29547f64718379356bd13263071a531bdd0896b5}{29547f6} & Courtney Bonn & Added Project 1 Folder\\\hline
\href{https://github.com/courtbonn/CS-444/commit/683a101800f08d0bcd4492d65d8829d25f488f21}{683a101} & Courtney Bonn & Revert "Added Project 1 Folder"\\\hline
\href{https://github.com/courtbonn/CS-444/commit/e794b6c0aad9a13a17177922b6ff415d677495b5}{e794b6c} & Anonymous & Project created\\\hline
\href{https://github.com/courtbonn/CS-444/commit/23bfd89f069e99232fe59b7730f7625cbf29fa44}{23bfd89} & Courtney LeeAnne Bonn & Update on Overleaf.\\\hline
\href{https://github.com/courtbonn/CS-444/commit/f9ef6fd2b923276b21d789c4caba0985800d1515}{f9ef6fd} & Courtney Bonn & Added Project 1 LaTex files\\\hline
\href{https://github.com/courtbonn/CS-444/commit/5d11dbfd983f4c3f1911da1797a2be0fae8bf2cf}{5d11dbf} & Courtney Bonn & Merge branch 'master' of https://git.overleaf.com/11476102qmhgtdkzzwrz\\\hline
\href{https://github.com/courtbonn/CS-444/commit/c96843d51fef0558071c08e6256c3020fd84a999}{c96843d} & Courtney LeeAnne Bonn & Update on Overleaf.\\\hline
\href{https://github.com/courtbonn/CS-444/commit/d7d8486bd31847448e59f0803e792d6ea53e516a}{d7d8486} & Courtney Bonn & Merge branch 'master' of https://git.overleaf.com/11476102qmhgtdkzzwrz\\\hline
\href{https://github.com/courtbonn/CS-444/commit/768a313b027c8be433190f3d92ee1d3f8d69daad}{768a313} & Courtney Bonn & removed overleaf folder\\\hline
\href{https://github.com/courtbonn/CS-444/commit/431243b6da2468a84fa6ef1b39b6490ea81620a0}{431243b} & Courtney Bonn & removed additional files\\\hline
\href{https://github.com/courtbonn/CS-444/commit/ab0f559ed2890ba06230c459c2b1b96a8f69ac6a}{ab0f559} & Courtney Bonn & Added Project 1 Folder\\\hline
\href{https://github.com/courtbonn/CS-444/commit/3f97e0af2ade6dd51b96d8d9cfd751fb70deff0d}{3f97e0a} & Courtney LeeAnne Bonn & Update on Overleaf.\\\hline
\href{https://github.com/courtbonn/CS-444/commit/91329b56b14c2eb1d1058f3f3275215813b83e6c}{91329b5} & Courtney Bonn & Merge branch 'master' of https://git.overleaf.com/11476102qmhgtdkzzwrz\\\hline
\href{https://github.com/courtbonn/CS-444/commit/8d79a3663a74c19792bf0bcaf63e81d6c8ed0202}{8d79a36} & Isaac Chan & add base concurrency assignment\\\hline
\href{https://github.com/courtbonn/CS-444/commit/2986ade21091b19b494a0ab684c209f17043f11c}{2986ade} & Courtney LeeAnne Bonn & Update on Overleaf.\\\hline
\href{https://github.com/courtbonn/CS-444/commit/aad3085e0ed73b8dc15ba16c44587b6216d605f4}{aad3085} & Courtney Bonn & Merge branch 'master' of https://git.overleaf.com/11476102qmhgtdkzzwrz\\\hline
\href{https://github.com/courtbonn/CS-444/commit/c91c8019f4ca95febb45ce0d5e767b08ca0d5edc}{c91c801} & Courtney Bonn & Removed overleaf files\\\hline
\href{https://github.com/courtbonn/CS-444/commit/83cdd06b6f86e488b8cb42351ff1ffe0f6f14bc9}{83cdd06} & Courtney Bonn & Added LaTex Overleaf File\\\hline
\href{https://github.com/courtbonn/CS-444/commit/26a943d14694a3da4a741850230e552ed686ffb1}{26a943d} & Isaac Chan & finalize RDRAND support\\\hline
\href{https://github.com/courtbonn/CS-444/commit/783432f7223bc4f0398a94de020115e8cbd18e8c}{783432f} & Isaac Chan & add todo notes for methods\\\hline
\href{https://github.com/courtbonn/CS-444/commit/fd6aa5dac056c23cfbb9cd182e1ae7190d0124a9}{fd6aa5d} & Courtney Bonn & added latex file\\\hline
\href{https://github.com/courtbonn/CS-444/commit/ae045f44596599c8b89804b6114574eeed76b664}{ae045f4} & Courtney Bonn & Updated hw1.tex\\\hline
\href{https://github.com/courtbonn/CS-444/commit/e35c2615d97f2625ebaa1e5845d799b49457d7b0}{e35c261} & Courtney Bonn & Updated hw1.tex\\\hline
\href{https://github.com/courtbonn/CS-444/commit/6de87295369d96c5261e6e729094806ad5c14014}{6de8729} & Courtney Bonn & Added pdf\\\hline
\href{https://github.com/courtbonn/CS-444/commit/30cc64929a330c8dea22b408f6f905ab82c5f9b8}{30cc649} & Courtney Bonn & deleted unnecessary folder\\\hline
\href{https://github.com/courtbonn/CS-444/commit/7567b06cd1bb85644d12c5c6f20e01439ac303b2}{7567b06} & Courtney Bonn & Moved files into correct folder\\\hline
\href{https://github.com/courtbonn/CS-444/commit/b27b47d9c1ae89a8396ace289ae123d3bdd30ea8}{b27b47d} & Courtney Bonn & deleted extra folder\\\hline
\href{https://github.com/courtbonn/CS-444/commit/65f44e9b825be2bfaf5fb62037531ac516fe91d3}{65f44e9} & Courtney Bonn & Added qemu flag explanations\\\hline
\href{https://github.com/courtbonn/CS-444/commit/0e5a2a8615574269544771707abf67ed5f775a46}{0e5a2a8} & Courtney Bonn & Added work log and bib\\\hline
\href{https://github.com/courtbonn/CS-444/commit/bc81e0c881560deeef263a0732dae8476999999d}{bc81e0c} & Courtney Bonn & updated PDF\\\hline
\href{https://github.com/courtbonn/CS-444/commit/28602d145c5249d1feb8a2813a429ddc43211c2c}{28602d1} & Isaac Chan & add mutexes and rand time to repo master\\\hline
\href{https://github.com/courtbonn/CS-444/commit/eba890e8ea9640c4723e587f8aedafa088e14bd9}{eba890e} & Isaac Chan & added the rest of hw1.c currently with seg fault\\\hline
\href{https://github.com/courtbonn/CS-444/commit/f64315399021183e4c4e9ddaa2b4f55bf24ca3f0}{f643153} & Courtney Bonn & fixed seg fault; program still not running correctly\\\hline
\href{https://github.com/courtbonn/CS-444/commit/223b3969975954043e29b636d3047910f5b09a26}{223b396} & Isaac Chan & add Isaac's worklog and part of concurrency write up.\\\hline
\href{https://github.com/courtbonn/CS-444/commit/59a039f3dc01a5f965dda35aebac82b56188ecc7}{59a039f} & Courtney Bonn & Added some lines to try and fix issue; not working still\\\hline
\href{https://github.com/courtbonn/CS-444/commit/a1571b350bf3104b5add44e6ce48917ca888dea6}{a1571b3} & Courtney Bonn & finished producer-consumer problem; no errors\\\hline
\href{https://github.com/courtbonn/CS-444/commit/cb9a9c988fdb22f5364a44ecfc95c18a3f72ae3d}{cb9a9c9} & Courtney Bonn & Added abstract, updated worklogs, added concurrency answers, fixed makefile\\\hline
\href{https://github.com/courtbonn/CS-444/commit/d2cd6dfd0888c09678fff0cb0b561ad5f65cfb9b}{d2cd6df} & Courtney Bonn & fixed pdf\\\hline\end{tabular}


\section{Work Log}
\begin{center}
\begin{tabular}{ c c c l }
 Date  & Time & Person & Event \\ \hline
                 
\end{tabular}
\end{center}

\section{Write Up}
\begin{enumerate}
                \item What do you think the main point of this assignment is?
                \item How did you personally approach the problem? Design decisions, algorithm, etc.
                \item How did you ensure your solution was correct? Testing details, for instance.
                \item What did you learn?
                \item How should the TA evaluate your work? Provide detailed steps to prove correctness 
\end{enumerate}

%\section*{Appendix 1: Source Code}
%\input{__mt19937ar.c.tex}

%\bibliographystyle{IEEEtran}
%	\bibliography{IEEEabrv,hw2bib}

\end{document}
