\documentclass[letterpaper,10pt,draftclsnofoot,onecolumn,titlepage]{IEEEtran}

\usepackage{graphicx}
\usepackage{amssymb}
\usepackage{amsmath}
\usepackage{amsthm}
\usepackage{alltt}
\usepackage{float}
\usepackage{color}
\usepackage{url}
\usepackage{enumitem}
%\usepackage{pstricks, pst-node}
\usepackage{geometry}
\usepackage{bookmark}

\geometry{margin = .75in}

\usepackage{hyperref}


\def\name{Courtney Bonn, Isaac Chan}
\def\grp{Group \#39}

\hypersetup{
	colorlinks = true,
	urlcolor = black,
	pdfauthor = {\name},
	pdftitle = {CS 444, Project 3},
	pdfsubject = {CS 444, Project 3},
	pdfpagemode = UseNone
}

\begin{document}

\title{Project 3: Encrypted Block Device}
\author{\name \\ \grp}

\maketitle

\begin{abstract}
   This document details our plan, worklog, and further documentation regarding the encrypted block device project.
\end{abstract}

\section{Design Plan}
Our plan for project 3 is to first get a basic ramdisk block driver up and running on the VM before we begin any encryption. 
We first read through the chapter on block drivers in \href{https://lwn.net/Kernel/LDD3/}{Linux Device Drivers} which provided a simple block driver called \textit{sbull}. 
Because this driver was based off an older version of the Linux Kernel, we did a bit more research for ramdisk drivers that might be more recent. 

This research led us to a comment on the book which led to a more recent version of the ramdisk driver. 
In a post by \href{http://blog.superpat.com/2010/05/04/a-simple-block-driver-for-linux-kernel-2-6-31/comment-page-2/#comment-148884}{Pat Patterson}, we found the ramdisk driver that would work with kernel 2.6.31. 
Because we are using kernel 3.19, we still weren't sure if this would work. 
In the comments of the post, which can be found at \url{http://blog.superpat.com/2010/05/04/a-simple-block-driver-for-linux-kernel-2-6-31/comment-page-2/#comment-148884}, there was a recent comment by \href{http://blog.superpat.com/2010/05/04/a-simple-block-driver-for-linux-kernel-2-6-31/comment-page-2/#comment-148884}{Sarge} that updated one line of the code to make it work for kernels 3.15 and later.  

With this base code, we were able to compile the kernel and boot the VM. 
Once in the VM, we were able to run a series of commands that loaded the ramdisk module, made a filesystem, mount the module, and then unmount and remove the module. The commands were:

	\begin{enumerate}
		\item scp (ONID Username)@os2.engr.oregonstate.edu:/scratch/fall2017/39/linux-ramdisk/drivers/block/ramd.ko .
		\item insmod ramd.ko
		\item fdisk /dev/sbd0

		\item Command: n
		\item Partition type: p
		\item Partition number: 1
		\item First sector: [press enter for default value]
		\item Last sector: [press enter for default value]

		\item Command: w

		\item mkfs /dev/sbd0p1
		\item mount /dev/sbd0p1 /mnt
		\item echo Hi \textgreater /mnt/file1
		\item cat /mnt/file1
		\item ls -l /mnt (just to view the file was created)
		\item umount /mnt
		\item rmmod ramd.ko 
	\end{enumerate}

Now that we were able to successfully run an unencrypted ramdisk, our next step was to begin the encryption section. 

Once the encryption step was done, we were able to check if the encryption was successful. After creating the \textit{file1} file, we can check in the raw device if there's anything that matches the \textit{file1} string with \textit{grep -a 'Hi' /sbd0p1} which should return nothing if it's being encrypted and decrypted correctly. After this we can unmount and remove the module as before.

\section{Version Control Log}

\section{Work Log}
\begin{center}
\begin{tabular}{p{3cm}p{1cm}p{1cm}p{10cm} }
 Date  & Time & Person & Event \\ \hline
November 1, 2017 & 9:20am & Courtney & Started HW3 LaTeX file \\
November 2, 2017 & 7:15pm & Courtney and Isaac & Begin researching Ram Disks \\
		 & 8:30pm & Courtney & Set up basic block driver \\
		 & 9:30pm & Isaac & Building and compiling ramd.c \\
November 3, 2017 & 9:00am & Courtney & Successfully boot VM and loaded the Ramd.c module, made the filesystem, mounted it, verified it worked, unmounted and removed module \\
November 5, 2017 & 1:30pm & Courtney & Began researching Crypto API, focusing on how to use module parameters for the key \\
November 8, 2017 & 2:00pm & Isaac & Start implementing crypto functionality \\
November 11, 2017 & 5:45pm & Courtney & Began testing crypto code \\
		  & 7:00pm & Isaac & Running crypto code \\
   		  & 8:45pm & Isaac & Finished crypto code \\
   		  & 8:45pm & Isaac & Wrapping up assignment documentation and start generating patch file \\
\end{tabular}
\end{center}

\section{Write Up}
\begin{enumerate}
                \item What do you think the main point of this assignment is? \\
The main point was to implement a block device in our virtual machine. The block device encrypted when read was called and decrypted when write was called. It replaced the existing RAM disk driver.
                \item How did you personally approach the problem? Design decisions, algorithm, etc. \\
We started with an block device without encryption and ensured that worked before attempting encryption. When beginning encryption, we looked into the Crypto API for methods we could utilize. Essentially the algorithm is as follows:
\begin{enumerate}
   \item Initiate the Crypto API
   \item Set the cipher key to either the default or the supplied argument
   \item Encrypt and decrypt one byte at a time, as reads and writes are called
   \item Free the cipher when finished
\end{enumerate}

                \item How did you ensure your solution was correct? Testing details, for instance. \\
Initially we had the block device print out each byte that was encrypted/decrpyted, but it proved to be too much to handle. We decided to create a file in the filesystem, then search for the string in the raw device. If it wasn't found, it means that the string is being encrypted correctly.
                \item What did you learn? \\
We learned a lot about block devices in Linux and how they handle IO. We also learned how to use the Crypto API.
\end{enumerate}

\section{Run Instructions}
To apply the patch:
\begin{enumerate}
   \item cd \$linux\_src/drivers/block
   \item patch $>$ (dir where this is)/linux.patch
\end{enumerate}

Then source the environment and start qemu. Then the block device can be run as described above.

\end{document}
