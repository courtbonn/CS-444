\documentclass[letterpaper,10pt,draftclsnofoot,onecolumn,titlepage]{IEEEtran}

\usepackage{graphicx}
\usepackage{amssymb}
\usepackage{amsmath}
\usepackage{amsthm}
\usepackage{alltt}
\usepackage{float}
\usepackage{color}
\usepackage{url}
\usepackage{enumitem}
%\usepackage{pstricks, pst-node}
\usepackage{geometry}
\usepackage{bookmark}

\geometry{margin = .75in}

\usepackage{hyperref}


\def\name{Courtney Bonn, Isaac Chan}
\def\grp{Group \#39}

\hypersetup{
	colorlinks = true,
	urlcolor = black,
	pdfauthor = {\name},
	pdftitle = {CS 444, Project 3},
	pdfsubject = {CS 444, Project 3},
	pdfpagemode = UseNone
}

\begin{document}

\title{Project 3: Encrypted Block Device}
\author{\name \\ \grp}

\maketitle

\begin{abstract}
\end{abstract}

\section{Design Plan}
Our plan for project 3 is to first get a basic ramdisk block driver up and running on the VM before we begin any encryption. 
We first read through the chapter on block drivers in \href{https://lwn.net/Kernel/LDD3/}{Linux Device Drivers} which provided a simple block driver called \textit{sbull}. 
Because this driver was based off an older version of the Linux Kernel, we did a bit more research for ramdisk drivers that might be more recent. 

This research led us to a comment on the book which led to a more recent version of the ramdisk driver. 
In a post by \href{http://blog.superpat.com/2010/05/04/a-simple-block-driver-for-linux-kernel-2-6-31/comment-page-2/#comment-148884}{Pat Patterson}, we found the ramdisk driver that would work with kernel 2.6.31. 
Because we are using kernel 3.19, we still weren't sure if this would work. 
In the comments of the post, which can be found at \url{http://blog.superpat.com/2010/05/04/a-simple-block-driver-for-linux-kernel-2-6-31/comment-page-2/#comment-148884}, there was a recent comment by \href{http://blog.superpat.com/2010/05/04/a-simple-block-driver-for-linux-kernel-2-6-31/comment-page-2/#comment-148884}{Sarge} that updated one line of the code to make it work for kernels 3.15 and later.  

With this base code, we were able to compile the kernel and boot the VM. 
Once in the VM, we were able to run a series of commands that loaded the ramdisk module, made a filesystem, mount the module, and then unmount and remove the module. The commands were:

	\begin{enumerate}
		\item scp (ONID Username)@os2.engr.oregonstate.edu:/scratch/fall2017/39/linux-ramdisk/drivers/block/ramd.ko .
		\item insmod ramd.ko
		\item fdisk /dev/sbd0

		\item Command: n
		\item Partition type: p
		\item Partition number: 1
		\item First sector: [press enter for default value]
		\item Last sector: [press enter for default value]

		\item Command: w

		\item mkfs /dev/sbd0p1
		\item mount /dev/sbd0p1 /mnt
		\item echo Hi \textgreater /mnt/file1
		\item cat /mnt/file1
		\item ls -l /mnt (just to view the file was created)
		\item umount /mnt
		\item rmmod ramd.ko 
	\end{enumerate}

Now that we were able to successfully run an unencrypted ramdisk, our next step was to begin the encryption section. 

\section{Version Control Log}

\section{Work Log}
\begin{center}
\begin{tabular}{p{3cm}p{1cm}p{1cm}p{10cm} }
 Date  & Time & Person & Event \\ \hline
November 1, 2017 & 9:20am & Courtney & Started HW3 LaTeX file \\
November 2, 2017 & 7:15pm & Courtney and Isaac & Begin researching Ram Disks \\
		 & 8:30pm & Courtney & Set up basic block driver \\
		 & 9:30pm & Isaac & Building and compiling ramd.c \\
November 3, 2017 & 9:00am & Courtney & Successfully boot VM and loaded the Ramd.c module, made the filesystem, mounted it, verified it worked, unmounted and removed module \\
November 5, 2017 & 1:30pm & Courtney & Began researching Crypto API, focusing on how to use module parameters for the key \\
\end{tabular}
\end{center}

\section{Write Up}
\begin{enumerate}
                \item What do you think the main point of this assignment is? \\
                \item How did you personally approach the problem? Design decisions, algorithm, etc. \\
                \item How did you ensure your solution was correct? Testing details, for instance.
                \item What did you learn?
\end{enumerate}



\end{document}
