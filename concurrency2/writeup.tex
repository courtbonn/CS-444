\documentclass[letterpaper,10pt,draftclsnofoot,onecolumn,titlepage]{IEEEtran}

\usepackage{graphicx}
\usepackage{amssymb}
\usepackage{amsmath}
\usepackage{amsthm}
\usepackage{alltt}
\usepackage{float}
\usepackage{color}
\usepackage{url}
\usepackage{enumitem}
%\usepackage{pstricks, pst-node}
\usepackage{geometry}
\usepackage{bookmark}

\geometry{margin = .75in}

\usepackage{hyperref}


\def\name{Courtney Bonn, Isaac Chan}
\def\grp{Group \#39}

\hypersetup{
	colorlinks = true,
	urlcolor = black,
	pdfauthor = {\name},
	pdftitle = {CS 444, Concurrency 2},
	pdfsubject = {CS 444, Concurrency 2},
	pdfpagemode = UseNone
}

\begin{document}

\title{Concurrency 2: The Dining Philosophers Problem}
\author{\name \\ \grp}

\maketitle


\section{ How should the TA evaluate your work? Provide detailed steps to prove correctness} 

There are two commands that will compile our concurrency file. 

\begin{enumerate}
	\item make con2
	\item gcc dining.c -pthread -lrt -o dining
\end{enumerate}

After the file is compiled, ./dining will run the program. The program will start and the philosophers will think and eat.
They will do this endlessly until the user interrupts the program. \\

\noindent We verified the correctness of the program by first printing the numbers of the philosophers instead of their names. This ensured that
we could manually check that there was no overlap of forks being used. Since there was no overlap, it meant that the fork mutexes were 
working correctly. We were also able to see that multiple philosophers were able to eat concurrently, as long as the forks they used were separate.\\

\noindent By having the program run endlessly, it shows that there is no deadlock occuring, when philosophers are trapped in an eat or think cycle with
no end, and not getting stuck and starving some philosophers. It also demonstrates that no philosophers were randomly dying through the program runtime. \\

\end{document}
