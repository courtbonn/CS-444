\documentclass[letterpaper,10pt,draftclsnofoot,onecolumn,titlepage]{IEEEtran}

\usepackage{graphicx}
\usepackage{amssymb}
\usepackage{amsmath}
\usepackage{amsthm}
\usepackage{alltt}
\usepackage{float}
\usepackage{color}
\usepackage{url}
\usepackage{enumitem}
%\usepackage{pstricks, pst-node}
\usepackage{geometry}
\usepackage{bookmark}

\geometry{margin = .75in}

\usepackage{hyperref}

\newcommand*{\signature}[1]{%
	\par\noindent\makebox[3.5in]{\hrulefill} \hfill\makebox[3.0in]{\hrulefill}%
	\par\noindent\makebox[3.5in][l]{#1}	    \hfill\makebox[3.0in][l]{Date}%
}%

\def\name{Courtney Bonn, Isaac Chan}
\def\grp{Group \#39}

\hypersetup{
	colorlinks = true,
	urlcolor = black,
	pdfauthor = {\name},
	pdftitle = {CS461 Problem Statement},
	pdfsubject = {CS461 Problem Statement},
	pdfpagemode = UseNone
}




%pull in the necessary preamble matter for pygments output
%\usepackage{fancyvrb}
\usepackage{color}
\usepackage[latin1]{inputenc}


\makeatletter
\def\PY@reset{\let\PY@it=\relax \let\PY@bf=\relax%
    \let\PY@ul=\relax \let\PY@tc=\relax%
    \let\PY@bc=\relax \let\PY@ff=\relax}
\def\PY@tok#1{\csname PY@tok@#1\endcsname}
\def\PY@toks#1+{\ifx\relax#1\empty\else%
    \PY@tok{#1}\expandafter\PY@toks\fi}
\def\PY@do#1{\PY@bc{\PY@tc{\PY@ul{%
    \PY@it{\PY@bf{\PY@ff{#1}}}}}}}
\def\PY#1#2{\PY@reset\PY@toks#1+\relax+\PY@do{#2}}

\expandafter\def\csname PY@tok@gd\endcsname{\def\PY@tc##1{\textcolor[rgb]{0.63,0.00,0.00}{##1}}}
\expandafter\def\csname PY@tok@gu\endcsname{\let\PY@bf=\textbf\def\PY@tc##1{\textcolor[rgb]{0.50,0.00,0.50}{##1}}}
\expandafter\def\csname PY@tok@gt\endcsname{\def\PY@tc##1{\textcolor[rgb]{0.00,0.25,0.82}{##1}}}
\expandafter\def\csname PY@tok@gs\endcsname{\let\PY@bf=\textbf}
\expandafter\def\csname PY@tok@gr\endcsname{\def\PY@tc##1{\textcolor[rgb]{1.00,0.00,0.00}{##1}}}
\expandafter\def\csname PY@tok@cm\endcsname{\let\PY@it=\textit\def\PY@tc##1{\textcolor[rgb]{0.25,0.50,0.50}{##1}}}
\expandafter\def\csname PY@tok@vg\endcsname{\def\PY@tc##1{\textcolor[rgb]{0.10,0.09,0.49}{##1}}}
\expandafter\def\csname PY@tok@m\endcsname{\def\PY@tc##1{\textcolor[rgb]{0.40,0.40,0.40}{##1}}}
\expandafter\def\csname PY@tok@mh\endcsname{\def\PY@tc##1{\textcolor[rgb]{0.40,0.40,0.40}{##1}}}
\expandafter\def\csname PY@tok@go\endcsname{\def\PY@tc##1{\textcolor[rgb]{0.50,0.50,0.50}{##1}}}
\expandafter\def\csname PY@tok@ge\endcsname{\let\PY@it=\textit}
\expandafter\def\csname PY@tok@vc\endcsname{\def\PY@tc##1{\textcolor[rgb]{0.10,0.09,0.49}{##1}}}
\expandafter\def\csname PY@tok@il\endcsname{\def\PY@tc##1{\textcolor[rgb]{0.40,0.40,0.40}{##1}}}
\expandafter\def\csname PY@tok@cs\endcsname{\let\PY@it=\textit\def\PY@tc##1{\textcolor[rgb]{0.25,0.50,0.50}{##1}}}
\expandafter\def\csname PY@tok@cp\endcsname{\def\PY@tc##1{\textcolor[rgb]{0.74,0.48,0.00}{##1}}}
\expandafter\def\csname PY@tok@gi\endcsname{\def\PY@tc##1{\textcolor[rgb]{0.00,0.63,0.00}{##1}}}
\expandafter\def\csname PY@tok@gh\endcsname{\let\PY@bf=\textbf\def\PY@tc##1{\textcolor[rgb]{0.00,0.00,0.50}{##1}}}
\expandafter\def\csname PY@tok@ni\endcsname{\let\PY@bf=\textbf\def\PY@tc##1{\textcolor[rgb]{0.60,0.60,0.60}{##1}}}
\expandafter\def\csname PY@tok@nl\endcsname{\def\PY@tc##1{\textcolor[rgb]{0.63,0.63,0.00}{##1}}}
\expandafter\def\csname PY@tok@nn\endcsname{\let\PY@bf=\textbf\def\PY@tc##1{\textcolor[rgb]{0.00,0.00,1.00}{##1}}}
\expandafter\def\csname PY@tok@no\endcsname{\def\PY@tc##1{\textcolor[rgb]{0.53,0.00,0.00}{##1}}}
\expandafter\def\csname PY@tok@na\endcsname{\def\PY@tc##1{\textcolor[rgb]{0.49,0.56,0.16}{##1}}}
\expandafter\def\csname PY@tok@nb\endcsname{\def\PY@tc##1{\textcolor[rgb]{0.00,0.50,0.00}{##1}}}
\expandafter\def\csname PY@tok@nc\endcsname{\let\PY@bf=\textbf\def\PY@tc##1{\textcolor[rgb]{0.00,0.00,1.00}{##1}}}
\expandafter\def\csname PY@tok@nd\endcsname{\def\PY@tc##1{\textcolor[rgb]{0.67,0.13,1.00}{##1}}}
\expandafter\def\csname PY@tok@ne\endcsname{\let\PY@bf=\textbf\def\PY@tc##1{\textcolor[rgb]{0.82,0.25,0.23}{##1}}}
\expandafter\def\csname PY@tok@nf\endcsname{\def\PY@tc##1{\textcolor[rgb]{0.00,0.00,1.00}{##1}}}
\expandafter\def\csname PY@tok@si\endcsname{\let\PY@bf=\textbf\def\PY@tc##1{\textcolor[rgb]{0.73,0.40,0.53}{##1}}}
\expandafter\def\csname PY@tok@s2\endcsname{\def\PY@tc##1{\textcolor[rgb]{0.73,0.13,0.13}{##1}}}
\expandafter\def\csname PY@tok@vi\endcsname{\def\PY@tc##1{\textcolor[rgb]{0.10,0.09,0.49}{##1}}}
\expandafter\def\csname PY@tok@nt\endcsname{\let\PY@bf=\textbf\def\PY@tc##1{\textcolor[rgb]{0.00,0.50,0.00}{##1}}}
\expandafter\def\csname PY@tok@nv\endcsname{\def\PY@tc##1{\textcolor[rgb]{0.10,0.09,0.49}{##1}}}
\expandafter\def\csname PY@tok@s1\endcsname{\def\PY@tc##1{\textcolor[rgb]{0.73,0.13,0.13}{##1}}}
\expandafter\def\csname PY@tok@sh\endcsname{\def\PY@tc##1{\textcolor[rgb]{0.73,0.13,0.13}{##1}}}
\expandafter\def\csname PY@tok@sc\endcsname{\def\PY@tc##1{\textcolor[rgb]{0.73,0.13,0.13}{##1}}}
\expandafter\def\csname PY@tok@sx\endcsname{\def\PY@tc##1{\textcolor[rgb]{0.00,0.50,0.00}{##1}}}
\expandafter\def\csname PY@tok@bp\endcsname{\def\PY@tc##1{\textcolor[rgb]{0.00,0.50,0.00}{##1}}}
\expandafter\def\csname PY@tok@c1\endcsname{\let\PY@it=\textit\def\PY@tc##1{\textcolor[rgb]{0.25,0.50,0.50}{##1}}}
\expandafter\def\csname PY@tok@kc\endcsname{\let\PY@bf=\textbf\def\PY@tc##1{\textcolor[rgb]{0.00,0.50,0.00}{##1}}}
\expandafter\def\csname PY@tok@c\endcsname{\let\PY@it=\textit\def\PY@tc##1{\textcolor[rgb]{0.25,0.50,0.50}{##1}}}
\expandafter\def\csname PY@tok@mf\endcsname{\def\PY@tc##1{\textcolor[rgb]{0.40,0.40,0.40}{##1}}}
\expandafter\def\csname PY@tok@err\endcsname{\def\PY@bc##1{\setlength{\fboxsep}{0pt}\fcolorbox[rgb]{1.00,0.00,0.00}{1,1,1}{\strut ##1}}}
\expandafter\def\csname PY@tok@kd\endcsname{\let\PY@bf=\textbf\def\PY@tc##1{\textcolor[rgb]{0.00,0.50,0.00}{##1}}}
\expandafter\def\csname PY@tok@ss\endcsname{\def\PY@tc##1{\textcolor[rgb]{0.10,0.09,0.49}{##1}}}
\expandafter\def\csname PY@tok@sr\endcsname{\def\PY@tc##1{\textcolor[rgb]{0.73,0.40,0.53}{##1}}}
\expandafter\def\csname PY@tok@mo\endcsname{\def\PY@tc##1{\textcolor[rgb]{0.40,0.40,0.40}{##1}}}
\expandafter\def\csname PY@tok@kn\endcsname{\let\PY@bf=\textbf\def\PY@tc##1{\textcolor[rgb]{0.00,0.50,0.00}{##1}}}
\expandafter\def\csname PY@tok@mi\endcsname{\def\PY@tc##1{\textcolor[rgb]{0.40,0.40,0.40}{##1}}}
\expandafter\def\csname PY@tok@gp\endcsname{\let\PY@bf=\textbf\def\PY@tc##1{\textcolor[rgb]{0.00,0.00,0.50}{##1}}}
\expandafter\def\csname PY@tok@o\endcsname{\def\PY@tc##1{\textcolor[rgb]{0.40,0.40,0.40}{##1}}}
\expandafter\def\csname PY@tok@kr\endcsname{\let\PY@bf=\textbf\def\PY@tc##1{\textcolor[rgb]{0.00,0.50,0.00}{##1}}}
\expandafter\def\csname PY@tok@s\endcsname{\def\PY@tc##1{\textcolor[rgb]{0.73,0.13,0.13}{##1}}}
\expandafter\def\csname PY@tok@kp\endcsname{\def\PY@tc##1{\textcolor[rgb]{0.00,0.50,0.00}{##1}}}
\expandafter\def\csname PY@tok@w\endcsname{\def\PY@tc##1{\textcolor[rgb]{0.73,0.73,0.73}{##1}}}
\expandafter\def\csname PY@tok@kt\endcsname{\def\PY@tc##1{\textcolor[rgb]{0.69,0.00,0.25}{##1}}}
\expandafter\def\csname PY@tok@ow\endcsname{\let\PY@bf=\textbf\def\PY@tc##1{\textcolor[rgb]{0.67,0.13,1.00}{##1}}}
\expandafter\def\csname PY@tok@sb\endcsname{\def\PY@tc##1{\textcolor[rgb]{0.73,0.13,0.13}{##1}}}
\expandafter\def\csname PY@tok@k\endcsname{\let\PY@bf=\textbf\def\PY@tc##1{\textcolor[rgb]{0.00,0.50,0.00}{##1}}}
\expandafter\def\csname PY@tok@se\endcsname{\let\PY@bf=\textbf\def\PY@tc##1{\textcolor[rgb]{0.73,0.40,0.13}{##1}}}
\expandafter\def\csname PY@tok@sd\endcsname{\let\PY@it=\textit\def\PY@tc##1{\textcolor[rgb]{0.73,0.13,0.13}{##1}}}

\def\PYZbs{\char`\\}
\def\PYZus{\char`\_}
\def\PYZob{\char`\{}
\def\PYZcb{\char`\}}
\def\PYZca{\char`\^}
\def\PYZam{\char`\&}
\def\PYZlt{\char`\<}
\def\PYZgt{\char`\>}
\def\PYZsh{\char`\#}
\def\PYZpc{\char`\%}
\def\PYZdl{\char`\$}
\def\PYZti{\char`\~}
% for compatibility with earlier versions
\def\PYZat{@}
\def\PYZlb{[}
\def\PYZrb{]}
\makeatother



\begin{document}

\title{Project 1: Getting Acquainted}
\author{\name \\ \grp}

\maketitle

\begin{abstract}
Abstract here.
\end{abstract}

\clearpage

%input the pygmentized output of mt19937ar.c, using a (hopefully) unique name
%this file only exists at compile time. Feel free to change that.


\section{Log of Commands}
\begin{enumerate}
\item cd /scratch/fall2017
\item mkdir 39
\item cd /scratch/fall2017/39
\item git clone git://git.yoctoproject.org/linux-yocto-3.19
\item cd linux-yocto-3.19
\item git status - to confirm we are on tag v3.19.2
\item cd ..
\item source /scratch/files/environment-setup-i586-poky-linux.csh
\item qemu-system-i386 -gdb tcp::5539 -S -nographic -kernel bzImage-qemux86.bin -drive file=core-image-lsb-sdk-qemux86.ext4,if=virtio -enable-kvm -net none -usb -localtime --no-reboot --append "root=/dev/vda rw console=ttyS0 debug"
\item gdb (in new terminal tab)
\item (gdb) target remote: 5539
\item (gdb) c
\item root (in VM)
\item cp /scratch/files/config-3.19.2-yocto-qemu /scratch/fall2017/39/linux-yocto-3.19/.config
\item make -j4 all
\item qemu-system-i386 -gdb tcp::5539 -S -nographic -kernel linux-yocto-3.19/arch/x86/boot/bzImage -drive file=core-image-lsb-sdk-qemux86.ext4,if=virtio -enable-kvm -net none -usb -localtime --no-reboot --append "root=/dev/vda rw console=ttyS0 debug"
\item gdb (in new terminal tab)
\item (gdb) target remote: 5539
\item (gdb) c
\item root (in VM)
\end{enumerate}

\section{Explanation of Qemu Flags}
To learn what the qemu flags represented, we researched the linux man page \cite{Qemu-kvm}.
\begin{itemize}
\item -gdb \\
	This enables the debug mode. 
\item tcp::5539 \\
	This specifies the port.  
\item -S \\
	This tells the CPU to not start right at startup. 
\item -nographic \\
	This disables graphics which makes qemu only display on the command line. 
\item -kernel \\
	This is where the kernel file location is defined. 
\item -drive file=<>,if=virtio \\
	This is where the file that will be used for the virtual disk is defined. 
\item -enable-kvm \\
	This enables KVM virtualization and is the reason the VM boots so quickly. 
\item -net none \\
	This says there should be no network devices configured. 
\item -usb \\
	This enables the USB driver. 
\item -localtime \\
	Set the CPU at local time. 
\item --no-reboot \\
	Don't reboot qemu, just exit. 
\item --append "root=/dev/vda rw console=ttyS0 debug" \\
	This tells qemu to launch in debug mode.
\end{itemize}

\section{Concurrency Write Up}
\begin{enumerate}  
		\item What do you think the main point of this assignment is?
		   The main point of the assignment is learn the basics of sharing a resource between multiple processes. The producer-consumer problem is a common problem in parallel processing.
		\item How did you personally approach the problem? Design decisions, algorithm, etc.
		   Essentially the problem is having multiple processes trying to write to one shared resource (the buffer). They can't alter the buffer at the same time so there must exist locks (mutexes) when the buffer is being used.
		   The consumer algorithm was as follows: Check for empty buffer, lock mutex, consume buffer item, unlock mutex, increment the semaphore spaces.
		   The producer algorithm was as follows: Check for full buffer, lock mutex, create item, add item to buffer, unlock mutex, increment the semaphore items. 
		\item How did you ensure your solution was correct? Testing details, for instance.
		   There are three main tests for correct producer-consumer solution implementation.
		   Producer and consumer don't alter the buffer at the same time.
		   If the buffer is empty, the consumer won't consume an event until there exists an event.
		   If the buffer is full, th
		\item What did you learn?
 	\end{enumerate}
    
\section{Version Control Log}

\section{Work Log}
\begin{center}
\begin{tabular}{ c c c c }
 Date  & Time & Person & Event \\ \hline
 October 7, 2017 & 7:10pm & Courtney & Set up LaTex template \\  
 				 & 7:30pm & Courtney & Set up Git repository \\
                 & 7:55pm & Courtney & Finished fixing issue with Github \\
                 & 8:55pm & Courtney & Tried setting up Overleaf with Github, unable to figure it out right now \\
                 & 9:01pm & Courtney & Tries running qemu command, gets error that says "qemu command not found" \\
                 & 9:39pm & Courtney & Resourced configuration file, successfully runs qemu command \\
                 & 9:42pm & Courtney & Opens new terminal and connects to gdb and remote port \\
                 & 9:45pm & Courtney & Successfully boots VM in qemu \\
                 & 10:09pm & Courtney & Booted VM using new kernel file \\
                 & 10:13pm & Courtney & Began adding command log to write up\\
                 & 11:00pm & Courtney & Added makefile and made sure it correctly build the tex file \\ \hline
 October 8, 2017 & 1:00pm & Courtney & Fixed the folders on Github to pull from Overleaf correctly \\
 				 & 6:45pm & Courtney & Began researching the qemu flags and adding explanations to the tex file \\
                 & 7:30pm & Courtney & Started compiling the work log \\
                 & 8:10pm & Courtney & Began reading 4.1 of Look Book of Semaphores \\ \hline
 October 5, 2017 & 4:00pm & Isaac & Set up shared directory \\
   		 & 4:30pm & Isaac & Run acl_open script to share \\
                 & 5:00pm & Isaac & Unsucessfully try to start the qemu VM \\
   		 & 5:30pm & Isaac & Sucessfully build the new kernel \\ \hline
 October 7, 2017 & 8:30pm & Isaac & Start the concurrency assignment \\
   		 & 10:30pm & Isaac & Set up assignment shell and add todo notes \\ \hline
 October 8, 2017 & 8:00pm & Isaac & Continue work on concurrency assignment
   		 & 9:00pm & Isaac & Add random number generation \\
   		 & 9:30pm & Isaac & Add threads and mutexes \\ \hline
 October 9, 2017 & 7:00pm & Isaac & Write the rest of the concurrency assignment \\
   		 & 8:00pm & Isaac & First version done, but seg fault \\
                 
                 
\end{tabular}
\end{center}

%\section*{Appendix 1: Source Code}
%\input{__mt19937ar.c.tex}

\bibliographystyle{IEEEtran}
	\bibliography{IEEEabrv,hw1bib}

\end{document}
