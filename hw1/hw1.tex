\documentclass[letterpaper,10pt,draftclsnofoot,onecolumn,titlepage]{IEEEtran}

\usepackage{graphicx}                                        
\usepackage{amssymb}                                         
\usepackage{amsmath}                                         
\usepackage{amsthm}                                          

\usepackage{alltt}                                           
\usepackage{float}
\usepackage{color}
\usepackage{url}

\usepackage{balance}
\usepackage[TABBOTCAP, tight]{subfigure}
\usepackage{enumitem}
\usepackage{pstricks, pst-node}

\usepackage{geometry}

\geometry{margin = .75in}
\geometry{textheight=8.5in, textwidth=6in}

%random comment

\newcommand{\cred}[1]{{\color{red}#1}}
\newcommand{\cblue}[1]{{\color{blue}#1}}

\newcommand{\toc}{\tableofcontents}

%\usepackage{hyperref}

\def\name{Courtney Bonn, Isaac Chan}
\def\grp{Group \#39}

%pull in the necessary preamble matter for pygments output
%\input{pygments.tex}

%% The following metadata will show up in the PDF properties
% \hypersetup{
%   colorlinks = false,
%   urlcolor = black,
%   pdfauthor = {\name},
%   pdfkeywords = {cs311 ``operating systems'' files filesystem I/O},
%   pdftitle = {CS 311 Project 1: UNIX File I/O},
%   pdfsubject = {CS 311 Project 1},
%   pdfpagemode = UseNone
% }

\parindent = 0.0 in
\parskip = 0.1 in

\begin{document}

\title{Project 1: Getting Acquainted}
\author{\name \\ \grp}

\maketitle

\begin{abstract}
Abstract here.
\end{abstract}

\clearpage

%input the pygmentized output of mt19937ar.c, using a (hopefully) unique name
%this file only exists at compile time. Feel free to change that.

\begin{itemize}

	\item An explanation of each and every flag in the listed qemu command-line
	\item Answer the following questions in sufficient detail (for the concurrency):
    \begin{enumerate}  
		\item What do you think the main point of this assignment is?
		\item How did you personally approach the problem? Design decisions, algorithm, etc.
		\item How did you ensure your solution was correct? Testing details, for instance.
		\item What did you learn?
 	\end{enumerate}
	\item Version control log (formatted as a table) -- there are any number of tools for generating a TeX table from repo logs
	\item Work log. What was done when? Be detailed.
\end{itemize}

\section{Log of Commands}
\begin{enumerate}
\item cd /scratch/fall2017
\item mkdir 39
\item cd /scratch/fall2017/39
\item git clone git://git.yoctoproject.org/linux-yocto-3.19
\item cd linux-yocto-3.19
\item git status - to confirm we are on tag v3.19.2
\item cd ..
\item source /scratch/files/environment-setup-i586-poky-linux.csh
\item qemu-system-i386 -gdb tcp::5539 -S -nographic -kernel bzImage-qemux86.bin -drive file=core-image-lsb-sdk-qemux86.ext4,if=virtio -enable-kvm -net none -usb -localtime --no-reboot --append "root=/dev/vda rw console=ttyS0 debug"
\item gdb (in new terminal tab)
\item (gdb) target remote: 5539
\item (gdb) c
\item root (in VM)
\item cp /scratch/files/config-3.19.2-yocto-qemu /scratch/fall2017/39/linux-yocto-3.19/.config
\item make -j4 all
\item qemu-system-i386 -gdb tcp::5539 -S -nographic -kernel linux-yocto-3.19/arch/x86/boot/bzImage -drive file=core-image-lsb-sdk-qemux86.ext4,if=virtio -enable-kvm -net none -usb -localtime --no-reboot --append "root=/dev/vda rw console=ttyS0 debug"
\item gdb (in new terminal tab)
\item (gdb) target remote: 5539
\item (gdb) c
\item root (in VM)


\end{enumerate}
%\section*{Appendix 1: Source Code}
%\input{__mt19937ar.c.tex}

\end{document}