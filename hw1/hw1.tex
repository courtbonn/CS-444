\documentclass[letterpaper,10pt,draftclsnofoot,onecolumn,titlepage]{IEEEtran}

\usepackage{graphicx}
\usepackage{amssymb}
\usepackage{amsmath}
\usepackage{amsthm}
\usepackage{alltt}
\usepackage{float}
\usepackage{color}
\usepackage{url}
\usepackage{enumitem}
%\usepackage{pstricks, pst-node}
\usepackage{geometry}
\usepackage{bookmark}

\geometry{margin = .75in}

\usepackage{hyperref}

\newcommand*{\signature}[1]{%
	\par\noindent\makebox[3.5in]{\hrulefill} \hfill\makebox[3.0in]{\hrulefill}%
	\par\noindent\makebox[3.5in][l]{#1}	    \hfill\makebox[3.0in][l]{Date}%
}%

\def\name{Courtney Bonn, Isaac Chan}
\def\grp{Group \#39}

\hypersetup{
	colorlinks = true,
	urlcolor = black,
	pdfauthor = {\name},
	pdftitle = {CS461 Problem Statement},
	pdfsubject = {CS461 Problem Statement},
	pdfpagemode = UseNone
}




%pull in the necessary preamble matter for pygments output
%\input{pygments.tex}


\begin{document}

\title{Project 1: Getting Acquainted}
\author{\name \\ \grp}

\maketitle

\begin{abstract}
In the first assignment for the term, we are tasked with making sure our tools work for the term. First, we work through getting the kernel up and running on the os2 server, using the provided files. Once we were successful running the kernel, we build a new kernel and ensure that boots the VM as well. After we successfully built a new kernel, we moved on finding a solution to the producer-consumer problem.  
\end{abstract}

\clearpage

%input the pygmentized output of mt19937ar.c, using a (hopefully) unique name
%this file only exists at compile time. Feel free to change that.


\section{Log of Commands}
\begin{enumerate}
\item cd /scratch/fall2017
\item mkdir 39
\item cd /scratch/fall2017/39
\item git clone git://git.yoctoproject.org/linux-yocto-3.19
\item cd linux-yocto-3.19
\item git status - to confirm we are on tag v3.19.2
\item cd ..
\item source /scratch/files/environment-setup-i586-poky-linux.csh
\item qemu-system-i386 -gdb tcp::5539 -S -nographic -kernel bzImage-qemux86.bin -drive file=core-image-lsb-sdk-qemux86.ext4,if=virtio -enable-kvm -net none -usb -localtime --no-reboot --append "root=/dev/vda rw console=ttyS0 debug"
\item gdb (in new terminal tab)
\item (gdb) target remote: 5539
\item (gdb) c
\item root (in VM)
\item cp /scratch/files/config-3.19.2-yocto-qemu /scratch/fall2017/39/linux-yocto-3.19/.config
\item make -j4 all
\item qemu-system-i386 -gdb tcp::5539 -S -nographic -kernel linux-yocto-3.19/arch/x86/boot/bzImage -drive file=core-image-lsb-sdk-qemux86.ext4,if=virtio -enable-kvm -net none -usb -localtime --no-reboot --append "root=/dev/vda rw console=ttyS0 debug"
\item gdb (in new terminal tab)
\item (gdb) target remote: 5539
\item (gdb) c
\item root (in VM)
\end{enumerate}

\section{Explanation of Qemu Flags}
To learn what the qemu flags represented, we researched the linux man page \cite{Qemu-kvm}.
\begin{itemize}
\item -gdb \\
	This enables the debug mode. 
\item tcp::5539 \\
	This specifies the port.  
\item -S \\
	This tells the CPU to not start right at startup. 
\item -nographic \\
	This disables graphics which makes qemu only display on the command line. 
\item -kernel \\
	This is where the kernel file location is defined. 
\item -drive file=<>,if=virtio \\
	This is where the file that will be used for the virtual disk is defined. 
\item -enable-kvm \\
	This enables KVM virtualization and is the reason the VM boots so quickly. 
\item -net none \\
	This says there should be no network devices configured. 
\item -usb \\
	This enables the USB driver. 
\item -localtime \\
	Set the CPU at local time. 
\item --no-reboot \\
	Don't reboot qemu, just exit. 
\item --append "root=/dev/vda rw console=ttyS0 debug" \\
	This tells qemu to launch in debug mode.
\end{itemize}

\section{Concurrency Write Up}
\begin{enumerate}  
		\item What do you think the main point of this assignment is?
		   The main point of the assignment is learn the basics of sharing a resource between multiple processes. The producer-consumer problem is a common problem in parallel processing.
		\item How did you personally approach the problem? Design decisions, algorithm, etc.
		   Essentially the problem is having multiple processes trying to write to one shared resource (the buffer). They can't alter the buffer at the same time so there must exist locks (mutexes) when the buffer is being used.
		   The consumer algorithm was as follows: Check for empty buffer, lock mutex, consume buffer item, unlock mutex, increment the semaphore spaces.
		   The producer algorithm was as follows: Check for full buffer, lock mutex, create item, add item to buffer, unlock mutex, increment the semaphore items. 
		\item How did you ensure your solution was correct? Testing details, for instance.
		   There are three main tests for correct producer-consumer solution implementation.
		   Producer and consumer don't alter the buffer at the same time.
		   The producer won't try to add an event if the buffer is full.
		   The consumer won't try to consume an event if the buffer is empty.

		   We tested our solution on the os server and outputted print statements on the state of the mutex. This demonstrated that it was unlocking and locking correctly so the producer and consumer weren't acting on the buffer at the same time. Then we were able to track the state of the buffer and saw that the consumer wasn't acting when it was empty and the producer wasn't acting when it was full.
		\item What did you learn?
		   We learned the basics of a kernel and running it in a VM. In the concurrency exercise, we learned a lot about semaphores and mutexes and working with p\_threads. We also learned how to embed ASM into C and got more practice debugging with GDB.
 	\end{enumerate}
    
\section{Version Control Log}
\begin{tabular}{l l l}\textbf{Detail} & \textbf{Author} & \textbf{Description}\\\hline
\href{https://github.com/courtbonn/CS-444/commit/29547f64718379356bd13263071a531bdd0896b5}{29547f6} & Courtney Bonn & Added Project 1 Folder\\\hline
\href{https://github.com/courtbonn/CS-444/commit/683a101800f08d0bcd4492d65d8829d25f488f21}{683a101} & Courtney Bonn & Revert "Added Project 1 Folder"\\\hline
\href{https://github.com/courtbonn/CS-444/commit/e794b6c0aad9a13a17177922b6ff415d677495b5}{e794b6c} & Anonymous & Project created\\\hline
\href{https://github.com/courtbonn/CS-444/commit/23bfd89f069e99232fe59b7730f7625cbf29fa44}{23bfd89} & Courtney LeeAnne Bonn & Update on Overleaf.\\\hline
\href{https://github.com/courtbonn/CS-444/commit/f9ef6fd2b923276b21d789c4caba0985800d1515}{f9ef6fd} & Courtney Bonn & Added Project 1 LaTex files\\\hline
\href{https://github.com/courtbonn/CS-444/commit/5d11dbfd983f4c3f1911da1797a2be0fae8bf2cf}{5d11dbf} & Courtney Bonn & Merge branch 'master' of https://git.overleaf.com/11476102qmhgtdkzzwrz\\\hline
\href{https://github.com/courtbonn/CS-444/commit/c96843d51fef0558071c08e6256c3020fd84a999}{c96843d} & Courtney LeeAnne Bonn & Update on Overleaf.\\\hline
\href{https://github.com/courtbonn/CS-444/commit/d7d8486bd31847448e59f0803e792d6ea53e516a}{d7d8486} & Courtney Bonn & Merge branch 'master' of https://git.overleaf.com/11476102qmhgtdkzzwrz\\\hline
\href{https://github.com/courtbonn/CS-444/commit/768a313b027c8be433190f3d92ee1d3f8d69daad}{768a313} & Courtney Bonn & removed overleaf folder\\\hline
\href{https://github.com/courtbonn/CS-444/commit/431243b6da2468a84fa6ef1b39b6490ea81620a0}{431243b} & Courtney Bonn & removed additional files\\\hline
\href{https://github.com/courtbonn/CS-444/commit/ab0f559ed2890ba06230c459c2b1b96a8f69ac6a}{ab0f559} & Courtney Bonn & Added Project 1 Folder\\\hline
\href{https://github.com/courtbonn/CS-444/commit/3f97e0af2ade6dd51b96d8d9cfd751fb70deff0d}{3f97e0a} & Courtney LeeAnne Bonn & Update on Overleaf.\\\hline
\href{https://github.com/courtbonn/CS-444/commit/91329b56b14c2eb1d1058f3f3275215813b83e6c}{91329b5} & Courtney Bonn & Merge branch 'master' of https://git.overleaf.com/11476102qmhgtdkzzwrz\\\hline
\href{https://github.com/courtbonn/CS-444/commit/8d79a3663a74c19792bf0bcaf63e81d6c8ed0202}{8d79a36} & Isaac Chan & add base concurrency assignment\\\hline
\href{https://github.com/courtbonn/CS-444/commit/2986ade21091b19b494a0ab684c209f17043f11c}{2986ade} & Courtney LeeAnne Bonn & Update on Overleaf.\\\hline
\href{https://github.com/courtbonn/CS-444/commit/aad3085e0ed73b8dc15ba16c44587b6216d605f4}{aad3085} & Courtney Bonn & Merge branch 'master' of https://git.overleaf.com/11476102qmhgtdkzzwrz\\\hline
\href{https://github.com/courtbonn/CS-444/commit/c91c8019f4ca95febb45ce0d5e767b08ca0d5edc}{c91c801} & Courtney Bonn & Removed overleaf files\\\hline
\href{https://github.com/courtbonn/CS-444/commit/83cdd06b6f86e488b8cb42351ff1ffe0f6f14bc9}{83cdd06} & Courtney Bonn & Added LaTex Overleaf File\\\hline
\href{https://github.com/courtbonn/CS-444/commit/26a943d14694a3da4a741850230e552ed686ffb1}{26a943d} & Isaac Chan & finalize RDRAND support\\\hline
\href{https://github.com/courtbonn/CS-444/commit/783432f7223bc4f0398a94de020115e8cbd18e8c}{783432f} & Isaac Chan & add todo notes for methods\\\hline
\href{https://github.com/courtbonn/CS-444/commit/fd6aa5dac056c23cfbb9cd182e1ae7190d0124a9}{fd6aa5d} & Courtney Bonn & added latex file\\\hline
\href{https://github.com/courtbonn/CS-444/commit/ae045f44596599c8b89804b6114574eeed76b664}{ae045f4} & Courtney Bonn & Updated hw1.tex\\\hline
\href{https://github.com/courtbonn/CS-444/commit/e35c2615d97f2625ebaa1e5845d799b49457d7b0}{e35c261} & Courtney Bonn & Updated hw1.tex\\\hline
\href{https://github.com/courtbonn/CS-444/commit/6de87295369d96c5261e6e729094806ad5c14014}{6de8729} & Courtney Bonn & Added pdf\\\hline
\href{https://github.com/courtbonn/CS-444/commit/30cc64929a330c8dea22b408f6f905ab82c5f9b8}{30cc649} & Courtney Bonn & deleted unnecessary folder\\\hline
\href{https://github.com/courtbonn/CS-444/commit/7567b06cd1bb85644d12c5c6f20e01439ac303b2}{7567b06} & Courtney Bonn & Moved files into correct folder\\\hline
\href{https://github.com/courtbonn/CS-444/commit/b27b47d9c1ae89a8396ace289ae123d3bdd30ea8}{b27b47d} & Courtney Bonn & deleted extra folder\\\hline
\href{https://github.com/courtbonn/CS-444/commit/65f44e9b825be2bfaf5fb62037531ac516fe91d3}{65f44e9} & Courtney Bonn & Added qemu flag explanations\\\hline
\href{https://github.com/courtbonn/CS-444/commit/0e5a2a8615574269544771707abf67ed5f775a46}{0e5a2a8} & Courtney Bonn & Added work log and bib\\\hline
\href{https://github.com/courtbonn/CS-444/commit/bc81e0c881560deeef263a0732dae8476999999d}{bc81e0c} & Courtney Bonn & updated PDF\\\hline
\href{https://github.com/courtbonn/CS-444/commit/28602d145c5249d1feb8a2813a429ddc43211c2c}{28602d1} & Isaac Chan & add mutexes and rand time to repo master\\\hline
\href{https://github.com/courtbonn/CS-444/commit/eba890e8ea9640c4723e587f8aedafa088e14bd9}{eba890e} & Isaac Chan & added the rest of hw1.c currently with seg fault\\\hline
\href{https://github.com/courtbonn/CS-444/commit/f64315399021183e4c4e9ddaa2b4f55bf24ca3f0}{f643153} & Courtney Bonn & fixed seg fault; program still not running correctly\\\hline
\href{https://github.com/courtbonn/CS-444/commit/223b3969975954043e29b636d3047910f5b09a26}{223b396} & Isaac Chan & add Isaac's worklog and part of concurrency write up.\\\hline
\href{https://github.com/courtbonn/CS-444/commit/59a039f3dc01a5f965dda35aebac82b56188ecc7}{59a039f} & Courtney Bonn & Added some lines to try and fix issue; not working still\\\hline
\href{https://github.com/courtbonn/CS-444/commit/a1571b350bf3104b5add44e6ce48917ca888dea6}{a1571b3} & Courtney Bonn & finished producer-consumer problem; no errors\\\hline
\href{https://github.com/courtbonn/CS-444/commit/cb9a9c988fdb22f5364a44ecfc95c18a3f72ae3d}{cb9a9c9} & Courtney Bonn & Added abstract, updated worklogs, added concurrency answers, fixed makefile\\\hline
\href{https://github.com/courtbonn/CS-444/commit/d2cd6dfd0888c09678fff0cb0b561ad5f65cfb9b}{d2cd6df} & Courtney Bonn & fixed pdf\\\hline\end{tabular}


\section{Work Log}
\begin{center}
\begin{tabular}{ c c c l }
 Date  & Time & Person & Event \\ \hline
 October 5, 2017 & 4:00pm & Isaac & Set up shared directory \\
                 & 4:30pm & Isaac & Run acl\_open script to share \\
                 & 5:00pm & Isaac & Unsucessfully try to start the qemu VM \\
                 & 5:30pm & Isaac & Sucessfully build the new kernel \\ \hline
 October 7, 2017 & 7:10pm & Courtney & Set up LaTex template \\  
 		 & 7:30pm & Courtney & Set up Git repository \\
                 & 7:55pm & Courtney & Finished fixing issue with Github \\
                 & 8:30pm & Isaac & Start the concurrency assignment \\
		 & 8:55pm & Courtney & Tried setting up Overleaf with Github, unable to figure it out right now \\
                 & 9:01pm & Courtney & Tries running qemu command, gets error that says "qemu command not found" \\
                 & 9:39pm & Courtney & Resourced configuration file, successfully runs qemu command \\
                 & 9:42pm & Courtney & Opens new terminal and connects to gdb and remote port \\
                 & 9:45pm & Courtney & Successfully boots VM in qemu \\
                 & 10:09pm & Courtney & Booted VM using new kernel file \\
                 & 10:13pm & Courtney & Began adding command log to write up\\
		 & 10:30pm & Isaac & Set up assignment shell and add todo notes \\
                 & 11:00pm & Courtney & Added makefile and made sure it correctly build the tex file \\ \hline
 October 8, 2017 & 1:00pm & Courtney & Fixed the folders on Github to pull from Overleaf correctly \\
 		 & 6:45pm & Courtney & Began researching the qemu flags and adding explanations to the tex file \\
                 & 7:30pm & Courtney & Started compiling the work log \\
                 & 8:00pm & Isaac & Continue work on concurrency assignment \\
		 & 8:10pm & Courtney & Began reading 4.1 of Look Book of Semaphores \\ 
		 & 9:00pm & Isaac & Add random number generation \\
                 & 9:30pm & Isaac & Add threads and mutexes \\ \hline
 October 9, 2017 & 5:30pm & Courtney & Continued researching P-C problem to try and finish the code \\
 		 & 7:00pm & Isaac & Write the rest of the concurrency assignment \\
	         & 8:00pm & Isaac & First version done, but seg fault \\ 
		 & 8:00pm & Courtney & Began debugging Seg Fault error on Concurrency exercise \\
                 & 8:31pm & Courtney & Fixed Seg fault; Program compiles with no errors or warnings and runs without a Seg fault, but doesn't ever finish executing \\
                 & 10:00pm & Courtney & After debugging for an hour and half, finally found the line that was the source of the problem (sem\_getvalue) \\
                 & 10:10pm & Courtney & Added counter to buffer and now code is entering the producer and consumer functions \\
		 & 10:40pm & Courtney & Verified everything is working correctly \\
		 & 10:50pm & Isaac & Began working on my work log and concurrency writeup\\
		 & 11:00pm & Courtney & Finished up my work log and wrote abstract\\
		 & 11:00pm & Isaac & Edited the makefile to include the C file\\
		 & 11:13pm & Courtney & Added Github version control log\\ \hline
		 
                 
                 
\end{tabular}
\end{center}

%\section*{Appendix 1: Source Code}
%\input{__mt19937ar.c.tex}

\bibliographystyle{IEEEtran}
	\bibliography{IEEEabrv,hw1bib}

\end{document}
