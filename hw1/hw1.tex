\documentclass[letterpaper,10pt,draftclsnofoot,onecolumn,titlepage]{IEEEtran}

\usepackage{graphicx}
\usepackage{amssymb}
\usepackage{amsmath}
\usepackage{amsthm}
\usepackage{alltt}
\usepackage{float}
\usepackage{color}
\usepackage{url}
\usepackage{enumitem}
%\usepackage{pstricks, pst-node}
\usepackage{geometry}
\usepackage{bookmark}

\geometry{margin = .75in}

\usepackage{hyperref}

\newcommand*{\signature}[1]{%
	\par\noindent\makebox[3.5in]{\hrulefill} \hfill\makebox[3.0in]{\hrulefill}%
	\par\noindent\makebox[3.5in][l]{#1}	    \hfill\makebox[3.0in][l]{Date}%
}%

\def\name{Courtney Bonn, Isaac Chan}
\def\grp{Group \#39}

\hypersetup{
	colorlinks = true,
	urlcolor = black,
	pdfauthor = {\name},
	pdftitle = {CS461 Problem Statement},
	pdfsubject = {CS461 Problem Statement},
	pdfpagemode = UseNone
}




%pull in the necessary preamble matter for pygments output
%\input{pygments.tex}


\begin{document}

\title{Project 1: Getting Acquainted}
\author{\name \\ \grp}

\maketitle

\begin{abstract}
Abstract here.
\end{abstract}

\clearpage

%input the pygmentized output of mt19937ar.c, using a (hopefully) unique name
%this file only exists at compile time. Feel free to change that.


\section{Log of Commands}
\begin{enumerate}
\item cd /scratch/fall2017
\item mkdir 39
\item cd /scratch/fall2017/39
\item git clone git://git.yoctoproject.org/linux-yocto-3.19
\item cd linux-yocto-3.19
\item git status - to confirm we are on tag v3.19.2
\item cd ..
\item source /scratch/files/environment-setup-i586-poky-linux.csh
\item qemu-system-i386 -gdb tcp::5539 -S -nographic -kernel bzImage-qemux86.bin -drive file=core-image-lsb-sdk-qemux86.ext4,if=virtio -enable-kvm -net none -usb -localtime --no-reboot --append "root=/dev/vda rw console=ttyS0 debug"
\item gdb (in new terminal tab)
\item (gdb) target remote: 5539
\item (gdb) c
\item root (in VM)
\item cp /scratch/files/config-3.19.2-yocto-qemu /scratch/fall2017/39/linux-yocto-3.19/.config
\item make -j4 all
\item qemu-system-i386 -gdb tcp::5539 -S -nographic -kernel linux-yocto-3.19/arch/x86/boot/bzImage -drive file=core-image-lsb-sdk-qemux86.ext4,if=virtio -enable-kvm -net none -usb -localtime --no-reboot --append "root=/dev/vda rw console=ttyS0 debug"
\item gdb (in new terminal tab)
\item (gdb) target remote: 5539
\item (gdb) c
\item root (in VM)
\end{enumerate}

\section{Explanation of Qemu Flags}
\begin{itemize}
\item -gdb \\
	This enables the debug mode. 
\item tcp::5539 \\
	This specifies the port.  
\item -S \\
	This tells the CPU to not start right at startup. 
\item -nographic \\
	This disables graphics which makes qemu only display on the command line. 
\item -kernel \\
	This is where the kernel file location is defined. 
\item -drive file=<> \\
	This is where the file that will be used for the virtual disk is defined. 
\item if=virtio
\item -enable-kvm \\
	This enables KVM virtualization and is the reason the VM boots so quickly. 
\item -net none \\
	This says there should be no network devices configured. 
\item -usb \\
	This enables the USB driver. 
\item -localtime \\
	Set the CPU at local time. 
\item --no-reboot \\
	Don't reboot qemu, just exit. 
\item --append "root=/dev/vda rw console=ttyS0 debug" \\
	This tells qemu to launch in debug mode.
\end{itemize}

\section{Concurrency Write Up}
\begin{enumerate}  
		\item What do you think the main point of this assignment is?
		\item How did you personally approach the problem? Design decisions, algorithm, etc.
		\item How did you ensure your solution was correct? Testing details, for instance.
		\item What did you learn?
 	\end{enumerate}
    
\section{Version Control Log}

\section{Work Log}

%\section*{Appendix 1: Source Code}
%\input{__mt19937ar.c.tex}

\end{document}